\def\year{2019}\relax
%File: formatting-instruction.tex
\documentclass[letterpaper]{article} %DO NOT CHANGE THIS
\usepackage{multirow}
\usepackage{amsmath}
\usepackage{amsfonts}
\newcommand{\Ep}{\mathbb{E}}
\newcommand{\Real}{\mathcal{R}}
\newcommand{\Rating}{\mathbf{X}}
\newcommand{\Loss}{\mathcal{L}}
\usepackage{aaai19}  %Required
\usepackage{times}  %Required
\usepackage{helvet}  %Required
\usepackage{courier}  %Required
\usepackage{url}  %Required
\usepackage{graphicx}  %Required
\frenchspacing  %Required
\setlength{\pdfpagewidth}{8.5in}  %Required
\setlength{\pdfpageheight}{11in}  %Required
%PDF Info Is Required:
  \pdfinfo{
/Title ()
/Author (AAAI Press Staff)}
\setcounter{secnumdepth}{0}  
 \begin{document}
% The file aaai.sty is the style file for AAAI Press 
% proceedings, working notes, and technical reports.
%
\title{Non-Compensatory Psychological Models for Recommender Systems}
\author{ID: 335
}
\maketitle
\begin{abstract}
The study of consumer psychology reveals two categories of consumption decision procedures: compensatory rules and non-compensatory rules. Existing recommendation models which are based on latent factor models assume the consumers follow the compensatory rules, i.e. the evaluations are based on a weighted or summated score over different aspects. However, it has been shown in the literature of consumer psychology that, consumers adopt non-compensatory rules more than compensatory rules. Our main contribution in this paper is to study the unexplored utilization of non-compensatory rules in recommendation models. 

Our general assumptions are (1) there are $K$ universal hidden aspects. In each evaluation session, only one aspect is chosen by the user as the prominent aspect. (2) Evaluations over different aspects are non-compensatory, i.e. the user has higher standards with respect to the item's performance on the prominent aspect and is more tolerant to the item's performance on other aspects. We show how these general assumptions can be applied to a wide range of existing recommender systems, including the point-wise rating prediction models such as conventional Matrix Factorization (MF), Neighborhood Factorization (AMF++), and low-rank matrix approximation (LLORMA), and the pair-wise ranking prediction models such as Bradley-Terry model (BTL) and Thurstonian model (e.g. BPR).  We experimentally show that adopting non-compensatory rules constantly improve performance of these rating  and ranking models on a variety of real-world recommendation data sets.
\end{abstract}



\section{Introduction}\label{sec:introduction}
The majority of state-of-the-art recommendation models are based on latent factor models. Generally, latent factor models transform both user preferences and item features into the same hidden feature spaces with $K$ aspects. To recover the observations (i.e. ratings or rankings) in any recommender system, they adopt the inner product of the user preferences and the item features. There are fruitful successful applications of latent factor models in  rating predictions~\cite{Koren2009Matrix,Koren2010Factor,Lee2014Local} and ranking reconstructions~\cite{Rendle2009BPR,Steck2015Gaussian,Zhao2018Factored,Shi2010List}.   

From the perspective of consumer decision making, all existing latent factor models fall into the category of compensatory rules. Consumers who adopt compensatory rules evaluate every product over multiple relevant aspects and compute a weighted or summated score for each product. Then they will select the product with the highest score. The result of such a procedure is  that a good performance on one aspect of a product compensates for poor performances on other aspects. 


However, in the study of human choice behavior, it is well regarded that there are two categories of decision making procedures, namely compensatory rules and non-compensatory rules. Furthermore, it is found in many surveys~\cite{Engel1986Consumer} that in most cases consumers make consumption related choices based on non-compensatory rules. Non-compensatory rules do not allow the shortcomings of a product to be balanced out by its attractive features. For example, the evaluation could be solely on the most important aspect and item's performance on other aspects does not affect the final judgement.

To the best of our knowledge, no previous work has been devoted to modeling non-compensatory rules in recommender systems. Our goal in this paper is to study this unexplored area. Our primary contribution is to model the assumptions in non-compensatory rules by (1) transforming user preferences and item features into the same $K-$dimensional hidden space as in most existing latent factor models, (2) picking a prominent aspect in each decision session, (3) and adopting different evaluation strategies on prominent and non-prominent aspects. We show that this strategy can be easily applied to a wide range of recommendation models, including point-wise rating prediction models such as the conventional Matrix Factorization (MF~\cite{Koren2009Matrix}), Matrix Factorization with neighborhood collaborative filtering (AMF~\cite{Koren2008Factorization}), and locally low-rank matrix approximation (LLORMA~\cite{Lee2013Local}) and pairwise ranking reconstruction models such as BTL model~\cite{Hu2016Improved} and BPR style  Thurstonian model~\cite{Rendle2009BPR}. We experimentally show that the non-compensatory versions of these models significantly outperform the original models.

The paper is organized as follows. In Sec.~\ref{sec:previousmodel} , we start with surveying the most commonly adopted latent factor models in the community of recommendation research. We categorize previous research work on the basis of  combinations of  different rating approximation formulas and  loss functions. In Sec.~\ref{sec:NCRmodel}, we describe our non-compensatory assumptions and develop non-compensatory versions of existing models. In Sec.~\ref{sec:experiment}, we experimentally show that the non-compensatory versions significantly outperform the original versions of existing models on a variety of real-world data sets. Finally, in Sec.~\ref{sec:conclusion} we conclude our work and future directions.

\section{Compensatory Recommendation Models}\label{sec:previousmodel}
In this section, we summarize and categorize recommendation models based on the rating prediction formulas and loss functions. We restrict our discussions to latent factor models, i.e. models where a universe of $K$ factors is used to project  user preferences and item features. Hereafter, unless stated otherwise, we use lower-case letters for indices, upper-case letters for universal constants, lower-case bold-face letters for vectors and upper-case bold-face letters for matrices. Specifically, $\mathbf{X}\in \Real^{M\times N}$ denotes the rating matrix, $\hat{\mathbf{X}}\in \Real^{M\times N}$ denotes the predicted rating matrix,  $\mathbf{p},\mathbf{q}\in \Real^K$ denotes the item features, $\mathbf{u}\in \Real^K$ denotes the user preferences.  

\subsection{Rating Prediction Formulas}
One goal of recommendation research is to recover the rating matrix $\Rating$, by minimizing a loss function $\Loss(\Rating,\hat{\Rating})$, which is usually defined as the regularized square loss between the predicted rating $\hat{\Rating}_{u,q}$ and the observed rating $\hat{\Rating}_{u,q}$ for each user $u$ who has rated item $q$. We list some of the most successful rating prediction formulas for $\hat{\Rating}$.

\textbf{Matrix Factorization.} In conventional matrix factorization~\cite{Koren2009Matrix}, the predicted rating can be computed as an inner product of user preferences and item features as follows.

\begin{equation}\label{equ:MF}
 \hat{\mathbf{X}}_{u,q}=\sum_{k=1}^{K} \mathbf{q}_k \mathbf{u}_k
\end{equation}

For simplicity we ignore the user specific or item specific bias~\cite{Koren2009Matrix}. A massive amount of techniques have been proposed based on Equ.~\ref{equ:MF}. Most of them modified the loss function ,e.g. by incorporating prior distributions over $\mathbf{p},\mathbf{u}$~\cite{salakhutdinov2008probabilistic}, adding priors over unknown values~\cite{Devooght2015Dynamic}, weighing different samples~\cite{Pil'aszy2010Fast} and so on.  

\textbf{Neighborhood Factorization.}  in traditional memory based collaborative filtering strategies, neighborhood information has been proved to be useful. It is possible  to embed such neighborhood information in latent factor models. Instead of directly modeling user preferences $\mathbf{u}$, each user is represented by items that he/she gives explicit or implicit feedback. For example, if we consider explicit feedback only, then each item is associated with two types of vectors $\mathbf{p},\mathbf{q}$, the rating prediction formula of AMF in ~\cite{Koren2008Factorization} is stated as follows.  
 \begin{equation}\label{equ:AMF}
\hat{\Rating}_{u,q}=\sum_{k=1}^{K} \mathbf{q}_{k} (\sum_{p \in R(u)} \mathbf{p}_k/\sqrt{|R(u)|} ),
\end{equation}

where  $R(u)$ is the set of rated items for $u$. AMF has been extended to SVD++~\cite{Koren2008Factorization} with implicit feedback. 

\textbf{Local Low-Rank Matrix Approximation.} The third type of rating prediction formula is  LLORMA~\cite{Lee2013Local}. The intuition of LLORMA is that the entire rating matrix $\Rating$ is not low-rank but a sub-matrix restricted to a neighborhood of similar users and items is low-rank.  Therefore, the predicted rating is aggregated over $S$ sub-matrices  as follows:

\begin{equation}\label{equ:LLORMA}
\hat{\Rating}_{u,q} = \sum_{t=1}^{S} \sum_k \mathbf{u}_{t, k} \frac{K((\mathbf{u}_t,\mathbf{i}_t),(\mathbf{u},\mathbf{q}))}{\sum_{s=1}^{S} K((\mathbf{u}_s,\mathbf{i}_s),(\mathbf{u},\mathbf{q}))} \mathbf{q}_{t,k}
\end{equation}

$\mathbf{u}_t, \mathbf{q}_t$ are the factorized user preferences and item features in the $t-$th sub-matrix,  $\mathbf{i}_s,\mathbf{i}_t$ are anchor points in the corresponding matrix to locate a neighborhood for low-rank decomposition, $K(\cdot)$ is a smoothing kernel. 


\subsection{Ranking Models}
Another goal of recommendation research is to reveal the observed rankings. We here consider pair-wise rankings $p\succ_u q$, where user $u$ prefers item $p$ over $q$. The pair-wise rankings can be generated from pre-processing  the ratings, i.e. $\Rating_{u,p}> \mu, \Rating_{u,q}<\mu$~\cite{Hu2017Decoupled}, or from explicit and implicit feedback, i.e. $\Rating_{u,p}\neq 0$ and $ \Rating_{u,q}$ doesn't exist~\cite{Rendle2009BPR}. 

A large body of previous research has been presented by employing a ranking aware loss function  $\Loss(p(p\succ_u q), o(p\succ_u q))$, where $p(p\succ_u q)$ is the predicted possibility and $o(p\succ_u q)$ is an indicator function of whether or not the ranking is observed. To generate the probability of pair-wise rankings $p(p\succ_u q)$,  each user-item combination is associated with a score, i.e. $\hat{\Rating}_{u,p},\hat{\Rating}_{u,q}$.  We list two most commonly adopted ranking models . 


\textbf{Thurstone Model} The most frequently adopted ranking model in recommendation systems is the Thurstone model~\cite{Thurstone1927law} which uses a non-linear transformation of the predicted ratings. 

\begin{equation}
p(p\succ_u q) = \frac{1} {1+\exp[-(\hat{\Rating}_{u,p}-\hat{\Rating}_{u,q})]}
\end{equation}

\textbf{Bradley-Terry Model.} The famous BTL model~\cite{Hunter2004MM} is extensively studied in learning to rank scenarios. BTL models the generation of ranking pairs by a division.

\begin{equation}\label{equ:BTL}
p(p\succ_u q) = \frac{\exp{\hat{\Rating}_{u,p}}}{\exp{\hat{\Rating}_{u,p}}+ \exp{\hat{\Rating}_{u,q}}} 
\end{equation}


In either ranking model, the score $\hat{\Rating}$ can be approximated by different ranking prediction formulas. We categorize existing ranking aware methods based on the combination of rating prediction formulas and the ranking models. 
\begin{itemize}
\item BTL model has been leveraged with MF prediction formula in~\cite{Hu2016Improved};
\item Thurstone model with standard matrix factorization prediction formula is first presented as BPR~\cite{Rendle2009BPR}, which maximizes the Bayesian posterior with respect to Thurstonian modeling of standard matrix factorization predictions. Abundant research has been carried out to improve BPR-style systems by modifying the sampling methods in optimization, including BTR++~\cite{Lerche2014Using}, WARP~\cite{Weston2011Wsabie}, DNS~\cite{Zhang2013Optimizing}, RankMBPR~\cite{Yu2016RankMBPR} and so on.
\item Thurstone model with  neighborhood factorized prediction formula AMF is first incorporated in a point-wise ranking framework In~\cite{Steck2015Gaussian},  FSBPR~\cite{Zhao2018Factored} implants AMF in a Thurstone model and maximizes its likelihood.
\item Thurstone model with local low-rank factorization prediction formula is utilized in LCR~\cite{Lee2014Local}.
 \end{itemize}

 



%\begin{table}[htp]\label{tab:survey}
%\caption{default}
%\begin{center}
%\begin{tabular}{|c|c|c|c|}
%\hline
% & \multicolumn{3}{c}{Rating}\\
% \cline{2-4}
%Ranking & MF & AMF & LLORMA \\\hline
%BTL &  & & \\\hline
%Thurstone & BPR\cite{Rendle2009BPR,Lerche2014Using,Weston2011Wsabie} & FSBPR\cite{Zhao2018Factored} & LCR~\cite{Lee2014Local} \\\hline
%\end{tabular}
%\end{center}
%\label{default}
%\end{table}%

The list is by no means exclusive. However, we believe that most of existing recommender systems are covered. It is worthy to point out that (1) we do not restrict the form of loss functions. For example, many ranking approaches consider Bayesian maximum posterior, cross entropy and other forms of loss functions. Nevertheless, the core ranking model is either BTL or Thurstone.   (2) Although we only study pair-wise ranking , the conclusion is insightful for other ranking-aware systems, i.e. point-wise and list-wise approaches. The reason is that, as shown in ~\cite{Steck2015Gaussian},  point-wise and list-wise loss functions can be decomposed to components which are directly based on each score $\hat\Rating_{u,p}$ and components that are not related to $\hat\Rating$. Thus our proposed strategy in Sec.~\ref{sec:NCRmodel} is also applicable to point-wise and list-wise ranking models.  

\section{Non-Compensatory Recommendation Models}\label{sec:NCRmodel}
We begin this section by reviewing the findings in consumer psychology study. Ever since the dawn of consumption psychology study, psychologists have been studying how consumers adopt different heuristics to  facilitate brand (or other consumption  related) choices. Two distinct categories of decision rules are found ~\cite{Engel1986Consumer}: compensatory rules and non compensatory rules. The decision rules can be naturally explained in the latent factor models. For example, compensatory rules are adopted if a consumer determines  options in terms of each factor and computes a weighted  or summated  score for each item, then selects the item that  scores the highest among  the alternatives evaluated. It is clear that all related work that has been described in previous section is the application of compensatory rules.

Non-compensatory rules include \textit{lexicographic, conjunction} and \textit{disjunction} rules. The conjunctive and disjunctive rules are often used in conjunction with lexicographic rules.

\begin{table}[htp]\label{tab:rules}
\caption{Illustrative example of non-compensatory rules}
\begin{center}
\begin{tabular}{|c|c|c|c|}
\hline
Item & Prominent aspect & \multicolumn{2}{|c|}{Not prominent aspects}\\\hline
$\mathbf{p}$ & 0.8 & 0.3 & 0.3  \\\hline
$\mathbf{q}$ & 0.7 & 0.2 & 0.2 \\\hline
$\mathbf{i}$ & 0.6 & 0.3 & 0.3 \\\hline
\end{tabular}
\end{center}
\label{default}
\end{table}%

We illustrate the non-compensatory rules using three toy item vectors $\mathbf{p,q,i}$  in Table.~\ref{tab:rules}. Under lexicographic rules, the user will first rank the factors, e.g. pick the prominent aspect according to user preference. Then items are evaluated on the prominent aspect, i.e. $p\succ_u q \succ_u i$.. Under conjunctive and disjunctive rules, the consumer imposes requirements for minimally acceptable values on each aspect separately.  Suppose the user's minimal acceptable value is $0.3$. If the user adopts conjunctive rules only, then the chosen items must be better than the minimal acceptable value on all aspects, i.e. $p.i$ will be  picked. If the user adopts disjunctive rules only, then the chosen items must satisfy the minimal acceptable value on at least one aspect, i.e. $p,q,i$ are all chosen. Finally, these rules an be combined. If the user adopts both lexicographic rules and conjunctive rules, then the result is $p\succ_u i$ and $i$ will not be chosen. 

We can see that non-compensatory rules differ from compensatory rules in two key points. (1) \textit{Distinguished factors}. In compensatory rules, different factors are essentially equivalent (i.e. all factors contribute to the final score), while in non-compensatory rules factors are not interchangeable (i.e. only the prominent factor is considered if there are no ties). (2) \textit{Distinguished evaluation metrics on each factor}. In compensatory rules, the evaluations on each factor follow the same framework (i.e. a product of user preference and item feature on the specific factor), while in non-compensatory rules, the evaluations on each factor are dissimilar (i.e. numerical comparisons on the prominent factor and acceptance/rejection on other factors).  

%Though these findings are insightful, we can not directly model them in the existing rating prediction or ranking reconstruction framework. 
For computational convenience, inspired by the psychological findings, we present the following two assumptions based on lexicographic and conjunction rules. (1) We assume that in each evaluation session\footnote{The evaluation session could be either a true user interaction session with multiple actions, or a pseudo session which contains one rating action. The impact of availability of session information is discussed in experiments. }, there is a prominent aspect. The choice of the prominent aspect is dependent on the user preferences. (2) We assume two types of evaluation strategies are adopted, one for the prominent aspect and the other for other non-prominent aspects. According to the above assumptions, we provide the non-compensatory versions of rating prediction formulas and ranking models.

 
\subsection{Non-Compensatory Rating Prediction Formulas}
Our goal here is to modify the rating prediction formulas as little as possible, while still preserving the most important properties of non-compensatory rules. Therefore, we follow the same notations for user preferences and item features. In each evaluation session, the hidden prominent aspect is sampled by $\frac{\exp \mathbf{u}_k}{\sum_{k'} \mathbf{u}_{k'}} $. We use a parameter $\theta$ to control the strength of prominent aspect, i.e. the evaluation on the prominent aspect is magnified by  $\exp \theta$. The prediction is generated across all possible hidden prominent aspects. This gives us the following non-compensatory versions of rating prediction formulas.

\textbf{Matrix Factorization: MF-NCR} 

\begin{equation}\label{equ:MF-NCR}
 \hat{\mathbf{X}}_{u,q}=\sum_{k=1}^{K} \frac{\exp \mathbf{u}_k}{\sum_{k'} \exp \mathbf{u}_{k'}} [ \exp\theta \mathbf{q}_k  + \sum_{k'\neq k} \mathbf{q}_{k'} ].
\end{equation}


\textbf{Neighborhood Factorization: AMF-NCR} implements a similar scheme by setting $u_k =\sum_{p \in R(u)} \mathbf{p}_k/\sqrt{|R(u)|} $, 

\begin{equation}\label{equ:AMF-NCR}
 \hat{\mathbf{X}}_{u,q}=\sum_{k=1}^{K} \frac{\exp (\sum_{p \in R(u)} \mathbf{p}_k )}{\sum_{k'} \exp  (\sum_{p \in R(u)} \mathbf{p}_{k'} ) } [ \exp\theta \mathbf{q}_k  + \sum_{k'\neq k} \mathbf{q}_{k'} ].
\end{equation}

\textbf{LLORMA-NCR} uses the same decomposition for each sub-matrix.  

\begin{eqnarray}\label{equ:LLORMA-NCR}
\hat{\Rating}_{u,q} = & \sum_{t=1}^{S} \sum_k  \frac{\exp \mathbf{u}_k}{\sum_{k'} \exp \mathbf{u}_{k'}}  \frac{K((\mathbf{u}_t,\mathbf{i}_t),(\mathbf{u},\mathbf{q}))}{\sum_{s=1}^{S} K((\mathbf{u}_s,\mathbf{i}_s),(\mathbf{u},\mathbf{q}))} \\\nonumber
& [ \exp\theta \mathbf{q}_{t,k}  + \sum_{k'\neq k} \mathbf{q}_{t,k'} ]
\end{eqnarray}

We can see that all these NCR versions are combinations of lexicographic and conjunction rules, where  $\exp\theta \rightarrow \infty$ indicates that the user adopts lexicographical rules only. 

\subsection{Non-Compensatory Ranking Models}

\textbf{Thurston-NCR}. The modification of Thurston model is straightforward, as the ranking probability involves a subtraction component of $\hat\Real_{u,q}$ which can be rreplaced by any NCR-version of rating prediction formulas. 

Inference of Thurston models is easily extensible. For example, if we use the Bayesian maximum posterior estimator as in BPR~\cite{Rendle2009BPR}, the loss function is defined as:
\begin{equation}
\Loss= -\sum_u \sum_{p\succ_u q}\ln \frac{1}{1+\exp{-[\hat\Rating_{u,p}-\hat\Rating_{u,q}]}}  - \lambda \|\Theta \|, 
\end{equation}

where  $\Theta$ is the set of all parameters. Thus the inference procedure is accomplished by stochastic gradient descent (SGD) with $\frac{\partial \Loss}{\partial \Theta}=  \sum_u \sum_{p\succ_u q} \frac{\partial \Loss}{\partial \Delta\hat{\Rating}_{u,p,q} } \frac{\partial \Delta\hat{\Rating}_{u,p,q}  }{\partial \Theta}$, where $\Delta\hat{\Rating}_{u,p,q} =\hat{\Rating}_{u,p}-\hat{\Rating}_{u,q}$. 



\textbf{BTL-NCR.} Finally we propose the non-compensatory version of BTL ranking model. In order to treat prominent and non-prominent aspects differently, we define the probability of any ranking pair $p\succ_u q$ as the product of results by factor-wise comparisons, based on a variant of BTL model with ties~\cite{Hunter2004MM}. Again, in each evaluation session, a hidden prominent aspect $k$ is sampled by user preference $\mathbf{u}$. The overall prediction is aggregated over all possible hidden prominent aspect $k$. 

\begin{equation}\label{equ:BTL-NCR}
p(p\succ_u q)  =  \prod_{k=1}^{K} \mathbf{u}_k [ {\frac{\mathbf{p}_k}{\mathbf{p}_k+\theta \mathbf{q}_k}}\prod_{k'\neq k}{ \frac{\theta \mathbf{p}_{k'}}{\mathbf{q}_{k'}+\theta \mathbf{p}_{k'}}}].
\end{equation}

where $\mathbf{u}_k >0, \sum_k \mathbf{u}_k=1$ and $\theta>1$.  BTL-NCR models the non-compensatory rules in a manner that (1) the evaluation is mainly based on the prominent aspect. The item $p$ is more likely to be preferred than $q$ by user $u$ if $p$ is significantly better than $q$ on the prominent aspect, i.e. $p_{k} > \theta q_{k}, \theta>1$. (2) The performance on other aspects are less  important. Because $p$ is considered to be as good as $q$, as long $\forall k'\neq k, \theta p_{k'} >  q_{k'}, \theta>1$.  BTL-NCR is also a combination of lexicographic rules and conjunction rules. An interpretation is that we dynamically set a minimal acceptance value for $p_{k'}$ on factor $k'\neq k$ based on the compared alternative $q_{k'}$, where the minimal acceptance value is $q_{k'}/\theta$.  The parameter $\theta$ controls the tolerance range. When $\theta \rightarrow \infty$, the users adopt lexicographic rules only.

To infer the parameters of BTL-NCR, we implement a stochastic expectation maximization (SEM) algorithm. In each E-step, we first draw the value of prominent aspect $k$ for each evaluation session by
 \begin{equation}
 k \sim u_k^{t} \frac{\mathbf{p}_{k}^t} {\mathbf{p}_{k}^t+\theta^t \mathbf{q}_{k}^t} \prod_{k'\neq k}  [\frac{\theta^t \mathbf{p}_{k'}^t} {\mathbf{q}_{k'}^t + \theta^t \mathbf{p}_{k'}^t}].
 \end{equation}
 where $t$ indicates the value obtained from the $t-$th round of SEM algorithm.
 In each M-step, we incorporate the MM bound in~\cite{Hunter2004MM} and maximize the log-likelihood of complete data. 




\section{Experiments}\label{sec:experiment}
We conduct experiments to evaluate the performance of non-compensatory rules in recommendation models. We conduct three sets of experiments on real world datasets. The first set of experiments is conducted to examine whether the NCR versions of rating prediction models outperform the original versions on rating data sets. 
The second set of experiments is conducted to examine whether NCR versions of ranking aware models outperform the original versions on data sets with implicit feedback.
The third set of experiments is conducted to examine whether NCR versions of ranking aware models outperform the original versions on data sets with user interaction session information.

\subsection{Rating Prediction Performance}

\textbf{Data Sets} We use the standard benchmarking datasets with user-item ratings. (1) Movielens (2) FilmTrust (3) CiaoDVD. The ratings are in the range of $1-5$ stars. Statistics of the datasets are described in Table.~\ref{tab:datasets}. For each dataset, we reserve users with at least $5$ ratings and randomly split training and test set by avoiding cold-start users and items. We consider each rating as an evaluation session. The reported results are averaged using 5-fold cross validation, 

\begin{table}[htp]
\caption{Statistics of Datasets with ratings and implicit feedback}
\begin{center}
\begin{tabular}{|c|c|c|c|c|}
\hline
Dataset & \#users & \#items & \#ratings & \#sessions \\\hline
Movielens & & & & \\\hline
FilmTrust & & & & \\\hline
CiaoDVD & & & & \\\hline
\end{tabular}
\end{center}
\label{tab:datasets}
\end{table}%
 

\textbf{Comparative Methods}. We compare the NCR improved versions with the original versions on three widely adopted rating prediction methods (1) MF~\cite{Koren2009Matrix}: standard matrix factorization  (2) AMF~\cite{Koren2008Factorization}: neighborhood factorization (3) LLORMA~\cite{Lee2013Local}: local low-rank matrix factorization. 

\textbf{Evaluation Metrics}.  The goal is to reconstruct the observed user-item ratings as accurate as possible. Hence we evaluate different approaches based on the following metrics. (1) AUC: (2) NDCG (3) RMSE (4) MAE (5) MRR

\begin{table}[htp]
\tiny
\caption{Comparative rating prediction performance}
\begin{center}
\begin{tabular}{|c|c|c|c|c|c|c|}
\hline
Dataset & Method & AUC & NDCG & RMSE & MAE & MRR \\\hline
\multirow{6}{*}{Movielens} & MF & & & & & \\\cline{2-7}
 & MF-NCR & & & & & \\\cline{2-7}
 & AMF &0.6043 &0.5003 & & &0.7506 \\\cline{2-7}
 & AMF-NCR &0.6129 &0.5027 & & &0.7622 \\\cline{2-7}
 & LLORMA & &0.8990 &0.9291 &0.7322 &0.5761 \\\cline{2-7}
 & LLORMA-NCR & &0.8994 &0.9217 &0.7288 &0.5761 \\\hline
\multirow{6}{*}{Filmtrust} & MF & & & & & \\\cline{2-7}
 & MF-NCR & & & & & \\\cline{2-7}
 & AMF &0.6244 &0.5055 & & &0.7622 \\\cline{2-7}
 & AMF-NCR &0.6436 &0.5098 & & &0.7717 \\\cline{2-7}
 & LLORMA & &0.8672 &0.8341 &0.6437 &0.6481 \\\cline{2-7}
 & LLORMA-NCR & &0.8684 &0.8310 &0.6417 &0.6533 \\\hline
 \multirow{6}{*}{CiaoDVD} & MF & & & & & \\\cline{2-7}
 & MF-NCR & & & & & \\\cline{2-7}
 & AMF &0.6211 &0.5048 & & &0.7607\\\cline{2-7}
 & AMF-NCR &0.7993 &0.5657 & & &0.8950 \\\cline{2-7}
 & LLORMA & &0.7827 &1.0436 &0.8188 &0.4883 \\\cline{2-7}
 & LLORMA-NCR & &0.7838 &1.0608 &0.8271 &0.4904 \\\hline
\end{tabular}
\end{center}
\label{tab:ratingresult}
\end{table}%

\subsection{Ranking Performance for Implicit Feedback}

\textbf{Comparative Methods}. We compare the NCR improved versions with the original versions on three widely adopted rating prediction methods (1) MF~\cite{Koren2009Matrix}: standard matrix factorization  (2) AMF~\cite{Koren2008Factorization}: neighborhood factorization (3) LLORMA~\cite{Lee2013Local}: local low-rank matrix factorization. 

\textbf{Evaluation Metrics}.  The goal is to reconstruct the observed user-item ratings as accurate as possible. Hence we evaluate different approaches based on the following metrics. (1) AUC: (2) NDCG (3) RMSE (4) MAE (5) MRR

\begin{table}[htp]
\tiny
\caption{Comparative ranking prediction performance with implicit feedback}
\begin{center}
\begin{tabular}{|c|c|c|c|c|c|c|}
\hline
Dataset & Method & MAP & NDCG & Prec & Recall & MRR \\\hline
\multirow{6}{*}{Movielens} & BTL & & & & & \\\cline{2-7}
 & BTL-NCR & & & & & \\\cline{2-7}
 & BPR & & & & & \\\cline{2-7}
 & BPR-NCR & & & & & \\\cline{2-7}
 & FSBPR & & & & & \\\cline{2-7}
  & FSBPR-NCR & & & & & \\\cline{2-7}
   & LCR & & & & & \\\cline{2-7}
 & LCR-NCR & & & & & \\\hline
\multirow{6}{*}{Filmtrust} & BTL & & & & & \\\cline{2-7}
 & BTL-NCR & & & & & \\\cline{2-7}
 & BPR & & & & & \\\cline{2-7}
 & BPR-NCR & & & & & \\\cline{2-7}
 & FSBPR & & & & & \\\cline{2-7}
  & FSBPR-NCR & & & & & \\\cline{2-7}
   & LCR & & & & & \\\cline{2-7}
 & LCR-NCR & & & & & \\\hline
\multirow{6}{*}{CiaoDVD} & BTL & & & & & \\\cline{2-7}
 & BTL-NCR & & & & & \\\cline{2-7}
 & BPR & & & & & \\\cline{2-7}
 & BPR-NCR & & & & & \\\cline{2-7}
 & FSBPR & & & & & \\\cline{2-7}
  & FSBPR-NCR & & & & & \\\cline{2-7}
   & LCR & & & & & \\\cline{2-7}
 & LCR-NCR & & & & & \\\hline
\end{tabular}
\end{center}
\label{tab:ratingresult}
\end{table}%




\subsection{Ranking Performance for Graded Sessional Feedback}

\textbf{Data Sets} In our model the prominent aspect is sampled for each evaluation session. In the previous experiments, the evaluation sessions are considered to be associated with each rating/click action. However, when the user interaction session information is available, the definition of evaluation session is different. We use two real world datasets with user-item interaction sessions. (1) Tmall   (2) FilmTrust (3) . 
\begin{table}[htp]
\caption{Statistics of Datasets with ratings and implicit feedback}
\begin{center}
\begin{tabular}{|c|c|c|c|c|}
\hline
Dataset & \#users & \#items & \#ratings & \#sessions \\\hline
Tmall-single & & & & \\\hline
Tmall-hybrid & & & & \\\hline
Yoochose & & & & \\\hline
\end{tabular}
\end{center}
\label{tab:datasets}
\end{table}%

\textbf{Comparative Methods}. We compare the NCR improved versions with the original versions on three widely adopted rating prediction methods (1) MF~\cite{Koren2009Matrix}: standard matrix factorization  (2) AMF~\cite{Koren2008Factorization}: neighborhood factorization (3) LLORMA~\cite{Lee2013Local}: local low-rank matrix factorization. 

\textbf{Evaluation Metrics}.  The goal is to reconstruct the observed user-item ratings as accurate as possible. Hence we evaluate different approaches based on the following metrics. (1) AUC: (2) NDCG (3) RMSE (4) MAE (5) MRR

\begin{table}[htp]
\tiny
\caption{Comparative ranking prediction performance with graded sessional feedback}
\begin{center}
\begin{tabular}{|c|c|c|c|c|c|c|}
\hline
Dataset & Method & MAP & NDCG & Prec & Recall & MRR \\\hline
\multirow{6}{*}{Tmall-single} & BTL &0.4348 &0.2814 &0.2787 &0.7253 &0.4890 \\\cline{2-7}
 & BTL-NCR &0.4408 &0.2853 &0.2810 &0.7308 &0.4973 \\\cline{2-7}
 & BPR & & & & & \\\cline{2-7}
 & BPR-NCR & & & & & \\\cline{2-7}
 & FSBPR &0.4163 &0.2732 &0.2734 &0.7092 &0.4717 \\\cline{2-7}
  & FSBPR-NCR &0.4193 &0.2747 &0.2749 &0.7130 &0.4740 \\\cline{2-7}
   & LCR & & & & & \\\cline{2-7}
 & LCR-NCR & & & & & \\\hline
\multirow{6}{*}{Tmall-hybrid} & BTL &0.5015 &0.3056 &0.2934 &0.7931 &0.5458 \\\cline{2-7}
 & BTL-NCR &0.5592 &0.3305 &0.3044 &0.8249 &0.6063 \\\cline{2-7}
 & BPR & & & & & \\\cline{2-7}
 & BPR-NCR & & & & & \\\cline{2-7}
 & FSBPR &0.4398 &0.2770 &0.2768 &0.7431 &0.4817 \\\cline{2-7}
  & FSBPR-NCR &0.4597 &0.2865 &0.2831 &0.7624 &0.5007 \\\cline{2-7}
   & LCR & & & & & \\\cline{2-7}
 & LCR-NCR & & & & & \\\hline
\multirow{6}{*}{Yoochoose} & BTL & & & & & \\\cline{2-7}
 & BTL-NCR & & & & & \\\cline{2-7}
 & BPR & & & & & \\\cline{2-7}
 & BPR-NCR & & & & & \\\cline{2-7}
 & FSBPR & & & & & \\\cline{2-7}
  & FSBPR-NCR & & & & & \\\cline{2-7}
   & LCR & & & & & \\\cline{2-7}
 & LCR-NCR & & & & & \\\hline
\end{tabular}
\end{center}
\label{tab:ratingresult}
\end{table}%


\subsection{Effect of Non-compensatory Rules}

\section{Related Work}\label{sec:relatedwork}

\section{Conclusion}\label{sec:conclusion}


\bibliography{/Users/linchen/Documents/GitHub/MyRef/reference.bib}
\bibliographystyle{aaai}
\end{document}
